\documentclass[letterpaper,12pt]{article}
\usepackage[spanish]{babel}
\spanishdecimal{.}
\selectlanguage{spanish}
\usepackage[spanish,onelanguage,ruled]{algorithm2e}
\usepackage[utf8]{inputenc}
\usepackage{graphicx}
\usepackage{caption}
\usepackage{subcaption}
\usepackage[top=2cm, bottom=2cm, left=2cm, right=2cm]{geometry}
\usepackage{hyperref}
\usepackage{verbatim}
\usepackage{amssymb}
\usepackage{mathtools}
\usepackage[tikz]{bclogo}
\newcommand\ddfrac[2]{\frac{\displaystyle #1}{\displaystyle #2}}
\DeclareMathOperator{\atantwo}{atan2}

\title{Práctica 10  \\ Teoría de Detección de Señales}
\author{Adriana Felisa Chávez de la Peña}
\date{Facultad de Psicología, UNAM \\Proyecto PAPIME 2016}
\begin{document}
\renewcommand{\tablename}{Tabla}
\maketitle
\section*{Objetivos}
\begin{itemize}
\item Aprender a detectar señales.
\item Aprender a detectar más señales.
\item Tener algo que entregar para el PAPIME para que nos vuelvan a dar presupuesto.
\end{itemize}

\section{Introducción}
Con frecuencia nos enfrentamos a situaciones en que debemos decidir si ``algo'' está o no ocurriendo para poder actuar en consecuencia, (e.g. ¿mi mamá está enojada?, ¿mi perro está enfermo?, ¿esta comida está pasada?). No parecería ser un gran problema si asumiéramos que somos infalibles en la detección de dichos casos, o bien, que aquello que nos interesa detectar es un evento tan particular que es completamente inconfundible con nada más en el mundo. Sin embargo, este casi nunca parece ser el caso: el mundo siempre está cargado de ruido e incertidumbre: tanto la información con base en la cual buscamos tomar una decisión, como la precisión con que nuestro sistema es capaz de evaluarla, son imperfectos. 
%%%%Hay dos dos puntos seguidos.
\begin{bclogo}[logo=\bccoeur,couleur=blue!10,arrondi=0.1,marge=10,barre=none]{Ejemplo}
Acabas de llegar a una fiesta y estás buscando a tu mejor amigo. Sabes cómo se ve y comienzas a buscar con la mirada a alguien que se le parezca, sin embargo, es muy probable que más de una persona comparta uno, o más, de sus rasgos, por lo que seguramente tendrás que mirar más de una vez antes de decidir que has encontrado a tu amigo y acercarte a saludarlo.
\begin{itemize}
\item Señal: tu mejor amigo.
\item Ruido: El resto de los invitados a la fiesta.
\end{itemize}
\end{bclogo}
La Teoría de Detección de Señales (TDS) constituye uno de los modelos estadísticos más sólidos y estudiados en psicología para describir y explicar el problema al que se enfrenta cualquier sistema ante este tipo de situaciones. En ella, se identifica como \textbf{señal} a aquel evento que se interesa detectar, y se define como \textbf{ruido} al resto de los estímulos con que coexiste y que pueden llegar a confundirse con la misma. 
%%Ampliamente amplio
La TDS ha sido ampliamente utilizada para describir un amplio número de fenómenos. Cuando hablamos de detección de señales, podemos referirnos a la señal tanto como un estímulo sensorial concreto (e.g. una luz, tono, u objeto particular), como una categoría más abstracta (e.g. una enfermedad, una emoción o un estado).  (Ver la sección de lecturas recomendadas para más ejemplos).

De manera muy general, se puede hablar de dos grandes supuestos en torno a los cuales se des arrolla la TDS:

1. Hay variabilidad, siempre (Ver Fig. 1).

a. Hay variabilidad en la señal

La idea central de variabilidad radica en la noción de que ningún estímulo se presenta ni se percibe exactamente igual cada vez que nos encontramos con él.  Es decir, cada vez que nos encontramos con la  señal en el mundo, ésta puede hacerlo dentro de un rango de posibilidades con cierta probabilidad. Esta idea se muestra gráficamente en la Figura 1, con la distribución normal azul identificada bajo la etiqueta de ‘Señal’. La idea es que la señal va adoptar una cierta forma de entre los puntos que abarca la distribución de probabilidad; siendo unas más probables que otras, conforme se aproximan a la media.

La variabilidad en la señal puede interpretarse en términos de dos fuentes: la percepción del sistema que ejecuta la tarea de detección, o la propia presentación estímulo en sí mismo. En el primer caso, se asume que cada vez que vemos un mismo estímulo que se mantiene constante en términos de sus propiedades físicas,  (e.g. una luz o un tono),  este puede ser percibido de manera distinta en cada presentación (i.e. unas veces parecerá un poco más intenso y otras, un poco menos). En el segundo caso, se asume que la señal puede tomar más de una forma, (e.g. si la señal es el enojo de un amigo, existen ciertos rasgos que son más o menos comúnmente asociados a su enfado; pero no siempre se va a ver exactamente igual).

b. Hay variabilidad en el entorno.

Por otro lado, es importante tomar en cuenta que las señales que interesa detectar coexisten en el mundo con otros estímulos; algunos de los cuales pueden llegar a producir una evidencia similar a la de nuestra señal y ser, por tanto, confundidos con la misma. Esta idea se representa en la Figura 1 con la distribución normal negra identificada bajo el nombre de ruido, que se traslapa con cierta probabilidad con la distribución de señal.

La Figura 1 corresponde a la representación gráfica asumida por la TDS para toda tarea de detección, e incorpora la noción de variabilidad previamente expuesta. 
\end{document}