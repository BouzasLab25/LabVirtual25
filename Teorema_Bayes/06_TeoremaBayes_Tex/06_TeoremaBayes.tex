\documentclass[letterpaper,12pt]{article}
\usepackage[spanish]{babel}
\spanishdecimal{.}
\usepackage[utf8]{inputenc}
\usepackage{graphicx}
\usepackage[top=2.5cm, bottom=2.5cm, left=2.5cm, right=2.5cm]{geometry}
\usepackage{hyperref}
\selectlanguage{spanish}
\usepackage[spanish,onelanguage,ruled]{algorithm2e}
\usepackage{caption}
\usepackage{subcaption}
\usepackage{verbatim}
\usepackage{amssymb}
\usepackage{mathtools}
\usepackage[tikz]{bclogo}
\newcommand\ddfrac[2]{\frac{\displaystyle #1}{\displaystyle #2}}

\title{Práctica 6 \\ Teorema de Bayes}
\author{Adriana Felisa Chávez de la Peña}
\date{Laboratorio 25\\Facultad de Psicología, UNAM\\Proyecto PAPIME 2016}
\begin{document}
\renewcommand{\tablename}{Tabla}
\maketitle

\section*{Objetivos}
\begin{itemize}
\item Volverse un experto en el Teorema de Bayes.
\item Volverse un experto programando el Teorema de Bayes.
\end{itemize}

\section{Introducción}
El \textbf{Teorema de Bayes} (también conocido como Regla de Bayes) constituye una herramienta útil, tan flexible como poderosa, para \textit{estimar la probabilidad de que un determinado evento ocurra dada la observación de cierta evidencia}.

La Regla de Bayes funciona a partir del cómputo de probabilidades. Como ya se discutió en el capítulo anterior, toda \textbf{Probabilidad} $p(x)$ se define como un número real entre 0 y 1 que representa el grado de certidumbre que se tiene sobre la ocurrencia de un evento $(x)$. En su definición más simple, la probabilidad puede computarse a partir de la razón entre el número de casos que incluyen al evento $x$  y el total de casos que es posible de observar.
\[p(x) = \ddfrac{\textrm{Casos contemplados en }x}{\textrm{Totalidad de casos posibles}}\]

\begin{bclogo}[logo=\bccoeur,couleur=blue!20]{Ejemplos}

\end{bclogo}

\end{document}

